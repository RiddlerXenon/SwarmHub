Рассматривается классическая дискретная по времени модель коллективного перемещения самодвижущихся частиц (модель Вичека), движение в которой каждой реализуется с постоянной по модулю скоростью и изменением направления для каждого шага на среднее по окрестности с добавлением некоторого значения аддитивного шума. Частицы являются точечными, пространственная область — квадрат $L\times L$ с периодическими граничными условиями. Взаимодействие является локальным и носит метрический характер: соседями считаются все частицы в круге радиуса $r$ вокруг текущей позиции. Эта минимальная постановка позволяет зафиксировать ключевую конкуренцию между выравниванием и шумом, а также приводит к кинетическому фазовому переходу между беспорядком и упорядоченным коллективным движением при изменении амплитуды шума или плотности $\rho=N/L^2$ [1].

Состояние частицы $i=1,\dots,N$ на шаге $n\in\mathbb N$ задаётся парой $(x_i^n,\theta_i^n)$, где позиция $x_i^n\in\mathbb R^2$, а скорость имеет фиксированный модуль $v_0>0$ и направление $\theta _i^n$
\begin{equation}
    v_i^n=v_0\bigl(\cos \theta _i^n,\ \sin\theta _i^n\bigr).
\end{equation}
Величины радиуса взаимодействия $r>0$, амплитуды шума $\eta\ge 0$, скорости $v_0$, шага по времени $\Delta t>0$ и размера $L$ являются параметрами модели. В данном контексте удобно оперировать также безразмерной длиной шага $\nu=v_0\Delta t$. Обновление выполняется синхронно: сначала вычисляются новые направления $\theta_i^{n+1}$ на основе конфигурации на шаге $n$, затем позиции $x_i^{n+1}$. Пусть $\mathcal N_i^n=\{\,j:\ \|x_j^n-x_i^n\|\le r\,\}$ — метрическое окружение частицы $i$ на шаге $n$. Тогда среднее направление по соседям определяется через аргумент векторной суммы единичных направляющих
\begin{equation}
    \bar\theta _i^n=\operatorname{Arg}\!\left(\sum_{j\in\mathcal N_i^n}e^{\mathrm i\theta _j^n}\right).
\end{equation}
Новая ориентация — сумма среднего и случайной угловой добавки $\Delta\theta_i^n$, равномерной на отрезке $[-\eta/2,\eta/2]$,
\begin{equation}
    \theta _i^{n+1}=\bar\theta _i^n+\Delta\theta _i^n,\qquad \Delta\theta _i^n\sim\mathcal U\!\left[-\tfrac{\eta}{2},\,\tfrac{\eta}{2}\right].
\end{equation}
Позиции затем сдвигаются на шаг вдоль нового направления
\begin{equation}
   x_i^{n+1}=x_i^n+v_0\bigl(\cos\theta _i^{n+1},\ \sin\theta _i^{n+1}\bigr)\,\Delta t, 
\end{equation}
а переход через границы реализуется периодизацией по сторонам квадрата (тор). Именно такая схема (средний азимут + равномерный угловой шум при фиксированной скорости) приводится в оригинальной работе [1], где в качестве единиц выбраны $r\equiv 1$ и $\Delta t\equiv 1$, так что $\nu=v_0$ есть безразмерная длина шага, а плотность $\rho$ и амплитуда шума $\eta$ служат основными управляющими параметрами. Рассматриваемый механизм интуитивно прост: каждая частица берёт курс на среднее направление ближайших, но слегка промахивается из-за шума, где при малом шуме локальные кластеры сливаются в глобально согласованное движение, при большом — направления разрушаются и остаются беспорядочными [1].

Глобальная упорядоченность измеряется нормированной средней скоростью (параметром порядка)
\begin{equation}
    \Phi^n=\frac{1}{Nv_0}\left\|\sum_{i=1}^N v_i^n\right\|\in[0,1],\qquad 
    \Phi=\langle \Phi^n\rangle_{n\gg 1}.
\end{equation}
При высоком шуме $\Phi\approx 0$ направления взаимно гасят друг друга, при низком — $\Phi$ становится строго положительной и в термодинамическом пределе стремится к величине порядка единицы. В оригинальной работе показано, что для фиксированных $\rho$ и $\nu$ существует критическое $\eta_c=\eta_c(\rho,\nu)$, при переходе через которое $\Phi$ обнуляется [1]. В численных экспериментах наблюдается непрерывное исчезновение порядка $\Phi\sim(\eta_c-\eta)^\beta$ с показателем $\beta\simeq 0{,}45$ для конечных систем, а сама перестройка носит кинетический, неравновесный характер — в отличие от модельной XY-ферромагнитной системы здесь сохраняется постоянная скорость частиц и отсутствует сохранение импульса взаимодействующих «спинов» [1].

С точки зрения вычислительной реализации один шаг алгоритма сводится к двум операциям: (i) поиску соседей в радиусе $r$ и вычислению аргумента суммы направляющих (переориентация) и (ii) сдвигу на фиксированную длину $\nu$ по новому курсу (перемещение) с последующей периодизацией по границам. При наивном поиске соседей сложность шага $O(N^2)$, что приемлемо для умеренных $N$. Стандартные же пространственные индексы (решетка, k-d-деревья) позволяют уменьшить константу и асимптотику без изменения модели. Роль шума является принципиальной: равномерная добавка к углу — внутренний или угловой шум — была исходным выбором [1]. Уже позднее изучались иные реализации (например, векторный шум до нормировки суммы направляющих), влияющие на характер перехода в больших системах [2].

\noindent\hrulefill

1. Vicsek, T., Czirók, A., Ben-Jacob, E., Cohen, I., & Shochet, O. (1995). Novel Type of Phase Transition in a System of Self-Driven Particles. Phys. Rev. Lett., 75(6), 1226–1229. https://doi.org/10.1103/PhysRevLett.75.1226

2. Grégoire G, Chaté H. Onset of collective and cohesive motion. Phys Rev Lett. 2004 Jan 16;92(2):025702. doi: 10.1103/PhysRevLett.92.025702. Epub 2004 Jan 15. PMID: 14753946.
