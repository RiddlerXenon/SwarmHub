alpha
Влияние феромона $\alpha$
Степень использования накопленного опыта в правиле выбора $p_{ij}^k(t)\propto [\tau_{ij}(t)]^{\alpha}[\eta_{ij}]^{\beta}$. Увеличение $\alpha$ усиливает детерминированность переходов к ребрам с большими $\tau$, сокращая исследование.

beta
Влияние эвристики $\beta$
Степень учёта априорной «желательности» $\eta_{ij}$ в $p_{ij}^k(t)$. При $\beta\to 0$ эвристика игнорируется. При больших $\beta$ выбор доминирует кратчайшими/наиболее выгодными локальными шагами.

rho !!! \in (0,1] !!!
Коэффициент испарения $\rho\in(0,1]$
Мера «забывания» в динамике $\tau_{ij}(t+1)=(1-\rho)\tau_{ij}(t)+\sum_k\Delta\tau_{ij}^k(t)$. Большие $\rho$ укорачивают память колонии и повышают адаптивность, а малые $\rho$ закрепляют найденные траектории.

Q !!! > 0 !!!
Интенсивность подкрепления $Q$
Масштаб откладываемого феромона $\Delta\tau_{ij}^k(t)=Q/L_k(t)$ на рёбрах решения. Линейно усиливает контраст между хорошими и плохими решениями. Влияет на скорость самоусиления доминирующих путей.

m
Численность колонии $m$
Количество независимых агентов. Увеличение снижает дисперсию оценки и ускоряет обнаружение качественных маршрутов при линейных вычислительных затратах.

T
Бюджет итераций $T$
Число глобальных циклов «решение–обновление». Прямо ограничивает время работы и глубину стабилизации распределения $\tau$.

tau0 !!! > 0 !!!
Начальная концентрация феромона $\tau_0$
Инициализационное значение $\tau_{ij}(0)=\tau_0$ на всех ребрах (дугах). Большие значения $\tau_0$ делают стартовое поведение ближе к равномерному, а малые усиливают роль $\eta$ на ранних шагах.

graphType
Тип графа (неориентированный/ориентированный)
Определяет симметрию феромонов: для неориентированного случая $\tau_{ij}=\tau_{ji}$ и $w_{ij}=w_{ji}$, для орграфа — независимые $\tau_{ij}$ и $\tau_{ji}$. Влияет на множество допустимых переходов и на нормировку $p_{ij}^k(t)$.

startDist
Распределение стартовых вершин
Закон выбора начальной вершины $i_0$ для каждого муравья: равномерно по $V$ либо по заданному распределению. Контролирует охват пространства решений на ранних итерациях.

seed
Инициализация ГПСЧ
Фиксация состояния ГПСЧ для воспроизводимости траекторий построения решений и последовательностей обновления $\tau$. Влияет на конкретную реализацию процесса.
