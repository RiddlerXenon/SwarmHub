**particleCount**
Число частиц $N$
Определяет размер популяции. При фиксированных $L$ и $r$ рост $N$ повышает плотность $\rho=N/L^2$ и, как правило, облегчает возникновение порядка.

**boxSize**
Размер области $L$ (квадрат $L\times L$, границы — периодические)
Масштаб всей сцены, определяющий вместе с $N$ плотность $\rho$. Расстояния для соседства считаются по тору (минимальный образ).

**densityMode**
Задание плотности
Выбор способа задания конфигурации:
— *N+L*: пользователь задаёт $N$ и $L$ напрямую;
— *$\rho$+L* (или *$\rho$+N*): один из параметров вычисляется из $\rho=N/L^2$.
Нужен, чтобы не давать пользователю избыточные степени свободы.

**interactionRadius**
Радиус взаимодействия $r>0$
Метрическое соседство $\mathcal N_i^n=\{j:\ \|x_j^n-x_i^n\|\le r\}$, Определяющее локальность выравнивания.

**noiseAmplitude**
Амплитуда углового шума $\eta\ge 0$
Равномерная добавка к углу после усреднения направлений. Увеличение $\eta$ разрушает порядок.

**speed**
Скорость частиц $v_0>0$
Постоянная по модулю скорость движения. Вместе с $\Delta t$ задаёт безразмерную длину шага $\nu=v_0\Delta t$ (полезно держать $\nu\lesssim r$ для более плавной динамики).

**timeStep**
Шаг по времени $\Delta t>0$
Дискретизация обновления. В классике зачастую полагается $\Delta t\equiv 1$. Если делается настраиваемым, то фактически управляет $\nu=v_0\Delta t$. Слишком крупный шаг визуально «рвёт» траектории.

**initHeadings**
Инициализация направлений $\theta_i^0$
Варианты: *uniform* $[0,2\pi)$; *aligned* (один общий курс); *cone($\sigma$)* (узкий конус вокруг заданного курса). Влияет на длину разогрева.

**initPositions**
Инициализация позиций $x_i^0$
Варианты: *uniform* (равномерно по тору), *grid+jitter* (решётка с шумом). Выбор влияет только на транзиент.

**burnIn**
Длина разогрева $B$ (число шагов до измерений)
Шаги, которые не учитываются в статистике $\Phi$, чтобы уйти от начальной конфигурации.

**avgWindow**
Окно усреднения $M$ (число шагов для оценки параметра порядка)
Используется для оценки $\Phi$ после разогрева.

**simSteps**
Длительность прогона $T$ (число шагов)
Общее число итераций симуляции. Должно быть $\ge B+M$ для корректной оценки $\widehat\Phi$.

**rngSeed**
Зерно генератора
Фиксирует выбор шума и инициализаций для воспроизводимости.

**vizTrails**
Отрисовка следов (длина хвоста)
Чисто визуальный параметр; на динамику не влияет, облегчает качественную оценку когерентности.